\documentclass[12pt,a4paper]{article}

\PassOptionsToPackage{pdftex}{graphicx}
\usepackage{graphicx}   
\usepackage[sfdefault,condensed]{cabin}       
\usepackage[T1]{fontenc}
\usepackage[utf8]{inputenc} 
\usepackage[brazil]{babel}
\usepackage{epstopdf}
\usepackage{soul}
\usepackage{array}
\usepackage[table]{xcolor}  % 'table' option loads »colortbl«
\usepackage{lipsum}
\usepackage{subfig}
\RequirePackage[numbers]{natbib}

\newcommand{\CEFETdisciplina}{Contexto Social e Profissional da Engenharia de Computação}
\newcommand{\CEFETprofessor}{Eduardo Gabriel Reis Miranda}



% aumentando as margens originais do documento
\usepackage{microtype} %melhor divisao de espaco entre letras/palavras
% aumentando as margens originais do documento
%\addtolength{\hoffset}{-1.5cm}
\addtolength{\textwidth}{2cm}
\addtolength{\hoffset}{-1cm}
\addtolength{\textheight}{-2cm}


%Para algoritmos ---------------------- 
\usepackage{listings}
\usepackage{color}
\definecolor{dkgreen}{rgb}{0,0.6,0}
\definecolor{gray}{rgb}{0.5,0.5,0.5}
\definecolor{mauve}{rgb}{0.58,0,0.82}
\definecolor{bg}{rgb}{0.95,0.95,0.95}
\lstset{ 
  language=Octave,                % the language of the code
  basicstyle=\footnotesize,       % the size of the fonts that are used for the code
  numbers=left,                   % where to put the line-numbers
  numberstyle=\tiny\color{gray},  % the style that is used for the line-numbers
  stepnumber=1,                   % the step between two line-numbers. If it's 1, each line 
                                  % will be numbered
  numbersep=5pt,                  % how far the line-numbers are from the code
  %backgroundcolor=\color{bg},    % choose the background color. You must add \usepackage{color}
  showspaces=false,               % show spaces adding particular underscores
  showstringspaces=false,         % underline spaces within strings
  showtabs=false,                 % show tabs within strings adding particular underscores
  frame=single,                   % adds a frame around the code
  rulecolor=\color{black},        % if not set, the frame-color may be changed on line-breaks within not-black text (e.g. commens (green here))
  tabsize=4,                      % sets default tabsize to 2 spaces
  captionpos=b,                   % sets the caption-position to bottom
  breaklines=true,                % sets automatic line breaking
  breakatwhitespace=false,        % sets if automatic breaks should only happen at whitespace
  title=\lstname,                 % show the filename of files included with \lstinputlisting;
                                  % also try caption instead of title
  keywordstyle=\color{blue}, %blue     % keyword style
  commentstyle=\color{dkgreen}, %dkgreen   % comment style
  stringstyle=\color{mauve}, %mauve      % string literal style
  escapeinside={\%*}{*)},         % if you want to add a comment within your code
  morekeywords={*,...}            % if you want to add more keywords to the set
}
%--------------------------------------
 

\usepackage{lastpage}
\newcommand*{\renameenviron}[1]{%
  \expandafter\let\csname exam-#1\expandafter\endcsname\csname #1\endcsname
  \expandafter\let\csname endexam-#1\expandafter\endcsname\csname end#1\endcsname
  \expandafter\let\csname #1\endcsname\relax
  \expandafter\let\csname end#1\endcsname\relax
}
\renameenviron{lhead}
\renameenviron{chead}
\renameenviron{rhead}
\renameenviron{lfoot}
\renameenviron{cfoot}
\renameenviron{rfoot}

% paragrafos
\setlength{\parskip}{.5cm}
\setlength{\parindent}{0pt}

\usepackage{fancyhdr}
\pagestyle{fancy}
\fancypagestyle{firstpage}{%
  \fancyhf{}%
  \fancyhead[L]{\includegraphics[width=3.6cm]{images/_logo.eps}\vspace{.3cm}}
  \fancyhead[R]{
	    \textbf{CENTRO FEDERAL DE EDUCAÇÃO TECNOLÓGICA DE 	    
	    MINAS GERAIS}\\CAMPUS V - UNIDADE DIVINÓPOLIS - Engenharia 
	    da Computação\\\vspace{.3cm}\CEFETdisciplina\\\CEFETprofessor\\\vspace{-.3cm}
  }
  \renewcommand{\headheight}{85pt}
}  
\fancypagestyle{otherpage}{%
  \fancyhf{}%
  \renewcommand{\footrulewidth}{1pt}
  \renewcommand{\headrulewidth}{1pt}
  \fancyhead[R]{\CEFETdisciplina\\
    \CEFETprofessor\\\vspace{-.3cm}
  }
  \renewcommand{\headheight}{45pt}
  \addtolength{\textheight}{50pt}
}  



\newcommand{\CEFETtitulo}[1]{
\begin{center}
\begin{Large}
\textbf{\\#1\\}
\end{Large}
\end{center}
\vspace{.5cm}
}

\newcommand{\CEFETsubtitulo}[1]{
\begin{center}
\begin{large}
\textbf{#1\\}
\end{large}
\end{center}
}


\newcommand{\CEFETopen}{
    \selectlanguage{brazil}
    \pagestyle{otherpage}
    \thispagestyle{firstpage}
}    
    
    
%para colunas--------------------------
\usepackage{multicol}
\setlength{\columnsep}{1cm}
\definecolor{cefetblue}{RGB}{2, 65, 112}
\makeatletter
\def\columnseprulecolor\vrule\@width\columnseprule{
%Separar colunas com pontinhos entre elas
%\vbox to \ht\mult@rightbox{\leaders\vbox{\kern2pt\hbox{.}\kern2pt}\vfill}}
%Separar colunas SEM pontinhos entre elas
\vbox to \ht\mult@rightbox{\leaders\vbox{\kern2pt\kern2pt}\vfill}}
\makeatother
\setlength{\columnseprule}{5px}   
%--------------------------------------    
    
     

\begin{document}
	\title{Modelo Relatório CEFET}
	\CEFETopen
    
    \CEFETtitulo{Trabalho Legal:\\Esse trabalho ficou muito legal}
    %Coloquem os nomes de todos os alunos aqui%
    Alunos: João Marques. Maria Marta. Marques João. Marta Maria


\section{Introdução}
\lipsum[1-2]
\subsection{Coisas}
\lipsum[1-2]

%importando algoritmo de arquivo de codigo
\vspace{.3cm}
\lstinputlisting[language=C++]{code.cpp}

\subsection{Outras Coisas}
\lipsum[1-2]

%Escrevendo algoritmo
\begin{lstlisting}[language=Java, caption={},label={}]
package pessoa;

public class Pessoa {
    public String nome;
    public int idade;

    public static void main(String[] args) {
        Pessoa p1 = new Pessoa(); 
    
    }
}
\end{lstlisting}



\section{Coisas Difíceis}
\lipsum[1]
\subsection{Coisas}

Isso é algo que \citet{Liu2007494} e \citet{law} disse em seu trabalho. A água é molhada \citep{1987element}. O sol é quente \citep{1987element,Liu2007494,site_sparselib}.

\lipsum[1-2]


%Tabela
\begin{center}
\vspace{.3cm}
\begin{tabular}{|l|l|l|}
\hline
\textbf{A}&\textbf{B}&\textbf{C}\\ \hline \hline
1&2&3\\
1&2&3\\
1&2&3\\ \hline
\end{tabular}
\vspace{.3cm}
\end{center}

\lipsum[1-2]

\subsection{Outras Coisas}
\lipsum[1-2]

De acordo com a Figura \ref{fig:figura2}.

\begin{figure}[h!]
\centering
\includegraphics[width=0.5\textwidth]{images/_logo}
\caption[]{Figura legal A}
\label{fig:figura1}
\end{figure}

\lipsum[1-2]

\begin{figure}[h]
  \centering
  \subfloat[]{\includegraphics[width=0.3\textwidth]{images/_logo}}\quad\quad
  \subfloat[]{\includegraphics[width=0.3\textwidth]{images/_logo}}
\caption[]{Figura legal B}
\label{fig:figura2}
\end{figure}

\lipsum[1-2]

\begin{eqnarray*}
(ano - nc) + (ano - ic) &\geq& 95\\
2\times ano - nc - ic  &\geq& 95\\
2\times ano &\geq& 95 + nc + ic\\
ano &\geq& (95 + nc + ic)/2
\end{eqnarray*}

\lipsum[1-2]
 
% Referências-------------------

% alpha - Sorted alphabetically. Labels are formed from name of author and year of publication. 
% plain - Sorted alphabetically. Labels are numeric. 
% unsrt - Like plain, but entries are in order of citation. 
% abbrv - Like plain, but more compact labels. 
\clearpage
%\bibliographystyle{abbrv}
%\bibliographystyle{plainnat-pt}
\bibliographystyle{plainnat}
\bibliography{lucaspa} %nome do arquivo bibtex

% -------------------------------

\end{document}
